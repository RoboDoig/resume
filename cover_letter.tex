%!TEX TS-program = xelatex
%!TEX encoding = UTF-8 Unicode
% ome CV LaTeX Template for CV/Resume
%
% The template for this document is by:
% Claud D. Park <posquit0.bj@gmail.com>
% https://github.com/posquit0/Awesome-CV
%
% Author:
% Andrew Erskine
%


%-------------------------------------------------------------------------------
% CONFIGURATIONS
%-------------------------------------------------------------------------------
% A4 paper size by default, use 'letterpaper' for US letter
%\documentclass[11pt, a4paper]{awesome-cv}
\documentclass[11pt, letterpaper]{awesome-cv}

% Configure page margins with geometry
\geometry{left=1.4cm, top=.8cm, right=1.4cm, bottom=1.8cm, footskip=.5cm}

% Specify the location of the included fonts
\fontdir[fonts/]

% Color for highlights
% Awesome Colors: awesome-emerald, awesome-skyblue, awesome-red, awesome-pink, awesome-orange
%                 awesome-nephritis, awesome-concrete, awesome-darknight
\colorlet{awesome}{awesome-red}
% Uncomment if you would like to specify your own color
% \definecolor{awesome}{HTML}{CA63A8}

% Colors for text
% Uncomment if you would like to specify your own color
% \definecolor{darktext}{HTML}{414141}
% \definecolor{text}{HTML}{333333}
% \definecolor{graytext}{HTML}{5D5D5D}
% \definecolor{lighttext}{HTML}{999999}

% Set false if you don't want to highlight section with awesome color
\setbool{acvSectionColorHighlight}{true}

% If you would like to change the social information separator from a pipe (|) to something else
\renewcommand{\acvHeaderSocialSep}{\quad\textbar\quad}


%-------------------------------------------------------------------------------
%	PERSONAL INFORMATION
%	Comment any of the lines below if they are not required
%-------------------------------------------------------------------------------
\name{Andrew}{Erskine}{}
\position{Senior Software Engineer}
\address{5 Hampden Road{\enskip\cdotp\enskip}London{\enskip\cdotp\enskip}United Kingdom}

\mobile{(+44) 7360109224} 
\email{erskine94@gmail.com}
% \homepage{william.weiskopf.me}
% \github{dartagan}
% \linkedin{whweiskopf}
% \stackoverflow{SO-id}{SO-name}
% \twitter{@twit}
% \skype{skype-id}
% \reddit{reddit-id}
% \extrainfo{extra informations}

% \quote{``If you want to build a ship, don't herd people together to collect wood and don't assign them tasks and work, but rather\\teach them to long for the endless immensity of the sea." -Antoine de Saint-Exup\'{e}ry}


%-------------------------------------------------------------------------------
\begin{document}

% Print the header with above personal informations
\makecvheader

% Print the footer with 3 arguments(<left>, <center>, <right>)
% Leave any of these blank if they are not needed
\makecvfooter
  {\today}
  {Andrew Erskine~~~·~~~Cover Letter}
  {\thepage}


%-------------------------------------------------------------------------------
%	CV/RESUME CONTENT
%	Each section is imported separately, open each file in turn to modify content
%-------------------------------------------------------------------------------

\vspace{5em}

Dear Zuckerman Institute team,

I was excited to learn of the current open position for a Senior Staff Associate focusing on development of software for neuroscience applications. I believe my extensive (10+ years) research experience in systems neuroscience (undergraduate, PhD, postdoctoral), as well as my current role as a software engineer developing open source tools (bonsai-rx ecosystem) for neuroscience research and analysis makes me an ideal candidate for the position.

In my current role as a Senior Software Engineer at NeuroGEARS I work as a contributor to the open-source bonsai-rx language and ecosystem. This platform is a popular tool within neuroscience research for building experimental control and data analysis pipelines and contains a suite of packages for applications in computer vision, machine learning, experimental design and many more. Through contributing to this platform I have gained substantial experience in best practices for developing and managing software projects:
\vspace{1em}
\begin{cvitems}
  \item {Git / GitHub version control approaches involving multiple collaborators across different teams}
  \item {Definition of coding standards and application via code review}
  \item {Continuous integration and unit testing of codebases to apply new features and fixes with limited disruption to existing users}
  \item {Generation of auto-documenting code, documentation websites, tutorials and workshops for effective dissemination of new features and libraries to end-users.}
\end{cvitems}
\vspace{1em}

Aside from my development work with NeuroGEARS, I am also heavily involved in workshops and training for users of bonsai-rx and have worked with many labs across the US (including in the Zuckerman Institute) and Europe to deploy bonsai-rx applications for their research requirements. As the user-base of bonsai-rx has grown I have also contributed to the community as a provider of technical support via the discussion forums.

In my previous role as a postdoctoral researcher in the Hires lab at USC I pursued a project using all-optical techniques to study information transfer between cortical layers, using 2-photon imaging and spatial-light modulation for optogenetic targeting. Working with these datasets required advanced data science skills and following effective practices for development and dissemination of data analysis code. During this work I was involved with:
\vspace{1em}
\begin{cvitems}
  \item {Management of git repositories for project data analysis and figure production (Python, MATLAB, jupyter, matplotlib)}
  \item {Utilising dimensionality-reduction techniques to examine evolution of neural population dynamics over task learning}
  \item {Deployment of machine-learning models (tensorflow, Keras) on cloud services (Google cloud) to increase analysis throughput in the lab}
  \item {Development of RNN models to infer functional connectivity in neural networks}
  \item {Organising and teaching workshops on best practices for version control (git) in research projects}
\end{cvitems}
\vspace{1em}

During my PhD research with Andreas Schaefer at the Francis Crick Institute I developed AutonoMouse: a high-throughput platform for mouse behavioural training. This necessitated building new software tools for flexible control of hardware in the system and integration for sensor data. To this end I developed a number of open-source projects including autonomouse-control (Python + PyQt GUI for designing, running and analysing long-term behavioural experiments) and PulseBoy (modular software platform for creating and synchronizing complex digital patterns to drive e.g. solenoid arrays).

Throughout these positions I have gained extensive experience in presentation of research and am highly proficient in visualization and communication of data. These skills have led to several publications in prominent journals (Nature, Neuron, eLife, PLOS ONE, JNeurosci). I am passionate about creating the tools that allow scientists to pursue their research goals effectively, and democratising those tools with dissemination and user-feedback. I believe that this goal, along with my considerable experience in neuroscience research and software engineering, would make me a valuable addition to the Zuckerman Institute team.  

Many thanks for your time and consideration. My resume, along with my personal websites (robodoig.github.io / andrewerskine.uk) include further details on my work, publications and personal projects. I hope to hear from you soon.

Best regards,

Andrew Erskine

%%% Points to hit
% Experience with building computational models of brain and/or behavior, including simulation, fitting, and experimental design.
% Experience analyzing neuroscientific experimental data.
% Proficiency in high-quality scientific/mathematical coding in more than one commonly used language, such as Python, C, C , Fortran, julia or MATLAB. Experience with HPC, GPU and/or code profiling and optimization a plus.
% Proficiency with git and GitHub or other version control frameworks.
% Proficiency with modern machine learning frameworks for autodifferentiation and GPU computing, preferably JAX and PyTorch.
% Knowledge of software engineering practices for working in groups, including software development life cycles, coding standards, code review and version control systems (git).
% Appropriate applied mathematical training (linear algebra, differential equations, etc.)
% Postdoctoral experience in  neuroscience, data science, computer science, engineering, mathematics or related technical discipline.
% Experience with open source communities and tools, especially in a research or educational context.
% Experience with community management and engagement, with a focus on teaching and knowledge sharing.
% Experience writing technical communication and documentation, and creating websites to host them.
% Experience with continuous integration or other software testing frameworks.
% Experience running software-oriented workshops or educational programs.
% How the applicant envisions their participation in the CCN scientific research environment.

% Blockchain technology is of interest to me: creating distributed ledgers without risk of trusting or storing in any single authority.  What is much more than a passing interest: Ethereum's smart contracts.  Finally, we can move beyond the risk of third parties.  Finally, applications can run without censorship or downtime.  It's such a powerful paradigm and I want to be a part of its early growth.  You don't need me to expound on the features of Ethereum though.  It is only important that you understand that I will spend my most valuable of resources pursuing this: my time and creativity.  I want to create the infrastructure that cities run on.  I want disrupt the systems people use every day, and to do it so smoothly that they can't imagine how it used to be.  I intend to pursue this program.

% A little bit about me: I love distributed systems and I geek out for version control. My friends listen to me wax poetically about the explicit beauty of Python, though I look forward to developing in newer languages and frameworks like Julia and React.  I dabble in DevOps and appreciate the importance of packaging. Listening is my absolute strength and I give only one form of attention: undivided.  That goes for individuals and towards projects. 

% Allow me to philosophize over questions of architecture, give me the latitude and freedom to explore ideas and I'll give you an exquisite system. Research, research, research - I prefer to synthesize different constructs into one whole, taking from the brilliant work of others or creating my own as needed; always through a thorough consideration of all components and their alternatives. A few bullets on me:

% \begin{itemize}[noitemsep]
%   \item Project driven
%   \item Elegant forms over minute details
%   %\item New project architecture over maintenance
%   \item Open source
%   \item Distributed systems
%   \item Proven best practices
% \end{itemize}

% I hope you will find me a good fit for this program,

% William

  
% % %-------------------------------------------------------------------------------
%	SECTION TITLE
%-------------------------------------------------------------------------------
\cvsection{Summary}


%-------------------------------------------------------------------------------
%	CONTENT
%-------------------------------------------------------------------------------
\begin{cventries}
    \vspace{2mm}
    \begin{cvparagraph}
        \begin{cvitems}
            \item{Proficient in quantitative methods including building data analysis pipelines for large, heterogeneous data sets and
            improving large codebases.}
            \vspace{1mm}
            \item{Exceptional engineering and science knowledge, including scientific communication tools (jupyter, matplotlib).}
            \vspace{1mm}
            \item{Dedication to writing clean, production-quality code and comfortable with unit testing, efficient version control, and common DevOps practices.}
            \vspace{1mm}
            \item{Experience working on multiplayer games and familiarity with common networking architectures and patterns.}
            \vspace{1mm}
            \item{Research-oriented software engineering skills, including fluency with libraries for scientific computing, deep learning
            and machine learning (Keras, TensorFlow, scikit-learn). Ability to rapidly acquire new technical knowledge and skills.}
            \vspace{1mm}
        \end{cvitems}
    \end{cvparagraph}

\end{cventries}

% %%-------------------------------------------------------------------------------
%	SECTION TITLE
%-------------------------------------------------------------------------------
\cvsection{Experience}


%-------------------------------------------------------------------------------
%	CONTENT
%-------------------------------------------------------------------------------
\begin{cventries}

%---------------------------------------------------------
  \cventry%
    {Senior Software Engineer} % Job title
    {NeuroGEARS Ltd.} % Organization
    {London, United Kingdom} % Location
    {April 2022 \-- Present} % Date(s)
    {
      \begin{cvsentence}Software / consulting company developing the Bonsai-Ex language and custom software tools for neuroscience research.\end{cvsentence}
      \begin{cvitems} % Description(s) of tasks/responsibilities
        \item {Improved and extended the popular Bonsai-Rx language for visual reactive programming.} \\
        Built and released Bonsai libraries for sensor interfaces (Tinkerforge), networking (ZeroMQ, Zyre, Lsl), streaming (FFmpeg) and Unity integration.
        \item {Developed Unity VR environments in collaboration with clients in research and industry.} \\
        With partners in clinical and sociological research, developed environments based on real-world city locations that participants can explore in VR while having key biophysical and attention signals monitored (heart-rate, galvanic skin response, eye-tracking, gaze fixation).
        \item {Worked with NeuroGEARS team to provide bespoke software tools to clients including user interfaces and documentation. Used reactive extensions in .NET and asynchronous programming to produce Bonsai-Rx workflows controlling complex neuroscience experiments. Communicated engineering process and requirements to non-technical clients.}
      \end{cvitems}
    }
    \begin{cventryskills}
      \item Bonsai
      \item C\# \& .NET
      \item Python
      \item MATLAB
      \item Unity
      \item ZeroMQ
      \item Avalonia
      \item Windows Forms
      \item MAUI
      \item Git
    \end{cventryskills}

%---------------------------------------------------------
  \cventry%
    {Postdoctoral Scholar} % Job title
    {The University of Southern California} % Organization
    {Los Angeles, California} % Location 
    {May 2018 \-- April 2022} % Date(s)
    {%
      \begin{cvsentence}Investigating somatosensory processing in neural circuits with 2p imaging and 3D optogenetics.\end{cvsentence}
      \begin{cvitems} % Description(s) of tasks/responsibilities
        \item {Reduced manual analysis time by modifying DeepLabCut for Google Cloud, allowing for fast, parallel usage on TB size whisker tracking datasets and speedup of data processing.}
        \item {Applied deep neural network models with dimensionality reduction methods to analyze neural population responses in high-dimensional space.}
        \item {Designed and deployed machine-learning pipelines to increase analysis throughput in the lab (Google Cloud, Colab).}
        \item {Mentored graduate and undergraduate students and provided training in data analysis and programming.}
        \item {Employed all-optical techniques to investigate neuronal ensemble recruitment in somatosensory cortex.}
      \end{cvitems}
    }
    \begin{cventryskills}
      \item Python
      \item MATLAB
      \item Keras
      \item tensorflow
      \item numpy
      \item pandas
      \item jupyter
    \end{cventryskills}

%---------------------------------------------------------
  \cventry%
    {PhD Student} % Job title
    {The Francis Crick Institute / University College London} % Organization
    {London, United Kingdom} % Location
    {September 2013 \-- May 2018} % Date(s)
    {%
      \begin{cvsentence}Building automated systems for mouse behavioral studies and investigating the temporal component of olfaction.\end{cvsentence}
      \begin{cvitems} % Description(s) of tasks/responsibilities
        \item {Redesigned high throughput mouse behavior system (AutonoMouse), that was based on an outdated software solution, using Python – including sensor interfaces, experiment control and database.}
        \item {Developed several auxiliary libraries that became standard tools: daqface for communicating with National Instruments ADCs and PulseBoy for designing complex digital command patterns.  }
        \item {Designed a novel odor-delivery device and software package for flexibly generating complex valve patterns with modular design (PulseBoy).}
      \end{cvitems}
    }
    \begin{cventryskills}
      \item Python
      \item MATLAB
      \item Qt
      \item nidaqmx
    \end{cventryskills}
\end{cventries}

% %%-------------------------------------------------------------------------------
%	SECTION TITLE
%-------------------------------------------------------------------------------
\cvsection{Education}


%-------------------------------------------------------------------------------
%	CONTENT
%-------------------------------------------------------------------------------
\begin{cventries}

%---------------------------------------------------------
  \cventry%
    {PhD in Neuroscience \--- Perception and representation of temporally patterned odor stimuli in the mammalian olfactory bulb} % Degree
    {University College London} % Institution
    {London, United Kingdom} % Location
    {2018} % Date(s)
    {%
    }

%---------------------------------------------------------
  \cventry%
    {MNeurosci, First Class Honours \---} % Degree
    {University of Manchester} % Institution
    {Manchester, United Kingdom} % Location
    {2013} % Date(s)
    {%
    }
    
%---------------------------------------------------------
\end{cventries}

% %%-------------------------------------------------------------------------------
%	SECTION TITLE
%-------------------------------------------------------------------------------
\cvsection{Projects}


%-------------------------------------------------------------------------------
%	CONTENT
%-------------------------------------------------------------------------------
\begin{cventries}

%---------------------------------------------------------
  \cventry%
    {} % Affiliation/role :: Project organization
    {\href{https://andrewerskine.uk}{andrewerskine.uk}} % Organization/group :: Project name
    {} % Location
    {} % Date(s)
    {Portfolio website featuring my personal projects in games, AI, UI and networking}
    \begin{cventryskills}
      \item Unity
      \item Blender
      \item DarkRift
      \item HTML, Javascript, CSS
      \item Multiplayer networking
      \item Computer vision
    \end{cventryskills}

%---------------------------------------------------------
\end{cventries}

% %\input{resume/honors.tex}
% %\input{resume/presentation.tex}
% %%-------------------------------------------------------------------------------
%	SECTION TITLE
%-------------------------------------------------------------------------------
\cvsection{Writing}


%-------------------------------------------------------------------------------
%	CONTENT
%-------------------------------------------------------------------------------
\begin{cventries}

%---------------------------------------------------------
  \cventry%
    {} % Affiliation/role :: Project organization
    {\href{https://dev.to/robodoig/a-practical-guide-to-rnns-for-neuroscience-research-in-keras-1ad2}{A practical guide to RNNs for neuroscience research in Keras}} % Organization/group :: Project name
    {} % Location
    {2021} % Date(s)
    {Article covering practical implementations and applications of recurrent neural network models for neuroscience research.}

  \cventry%
    {} % Affiliation/role :: Project organization
    {\href{https://dev.to/robodoig/unity-multiplayer-bottom-to-top-46cj}{Unity multiplayer: bottom to top}} % Organization/group :: Project name
    {} % Location
    {2021} % Date(s)
    {Tutorial outlining the development of a full-stack multiplayer Unity app. Now featured as part of the official documentation for the DarkRift multiplayer framework.}
%---------------------------------------------------------
\end{cventries}

% %\input{resume/committees.tex}


%-------------------------------------------------------------------------------
\end{document}
