%!TEX TS-program = xelatex
%!TEX encoding = UTF-8 Unicode
% ome CV LaTeX Template for CV/Resume
%
% The template for this document is by:
% Claud D. Park <posquit0.bj@gmail.com>
% https://github.com/posquit0/Awesome-CV
%
% Author:
% Andrew Erskine
%


%-------------------------------------------------------------------------------
% CONFIGURATIONS
%-------------------------------------------------------------------------------
% A4 paper size by default, use 'letterpaper' for US letter
%\documentclass[11pt, a4paper]{awesome-cv}
\documentclass[11pt, letterpaper]{awesome-cv}

% Configure page margins with geometry
\geometry{left=1.4cm, top=.8cm, right=1.4cm, bottom=1.8cm, footskip=.5cm}

% Specify the location of the included fonts
\fontdir[fonts/]

% Color for highlights
% Awesome Colors: awesome-emerald, awesome-skyblue, awesome-red, awesome-pink, awesome-orange
%                 awesome-nephritis, awesome-concrete, awesome-darknight
\colorlet{awesome}{awesome-red}
% Uncomment if you would like to specify your own color
% \definecolor{awesome}{HTML}{CA63A8}

% Colors for text
% Uncomment if you would like to specify your own color
% \definecolor{darktext}{HTML}{414141}
% \definecolor{text}{HTML}{333333}
% \definecolor{graytext}{HTML}{5D5D5D}
% \definecolor{lighttext}{HTML}{999999}

% Set false if you don't want to highlight section with awesome color
\setbool{acvSectionColorHighlight}{true}

% If you would like to change the social information separator from a pipe (|) to something else
\renewcommand{\acvHeaderSocialSep}{\quad\textbar\quad}


%-------------------------------------------------------------------------------
%	PERSONAL INFORMATION
%	Comment any of the lines below if they are not required
%-------------------------------------------------------------------------------
\name{Andrew}{Erskine}{}
\position{Senior Software Engineer}
\address{5 Hampden Road{\enskip\cdotp\enskip}London{\enskip\cdotp\enskip}United Kingdom}

\mobile{(+44) 7360109224} 
\email{erskine94@gmail.com}
% \homepage{william.weiskopf.me}
% \github{dartagan}
% \linkedin{whweiskopf}
% \stackoverflow{SO-id}{SO-name}
% \twitter{@twit}
% \skype{skype-id}
% \reddit{reddit-id}
% \extrainfo{extra informations}

% \quote{``If you want to build a ship, don't herd people together to collect wood and don't assign them tasks and work, but rather\\teach them to long for the endless immensity of the sea." -Antoine de Saint-Exup\'{e}ry}


%-------------------------------------------------------------------------------
\begin{document}

% Print the header with above personal informations
\makecvheader

% Print the footer with 3 arguments(<left>, <center>, <right>)
% Leave any of these blank if they are not needed
\makecvfooter
  {\today}
  {Andrew Erskine~~~·~~~Cover Letter}
  {\thepage}


%-------------------------------------------------------------------------------
%	CV/RESUME CONTENT
%	Each section is imported separately, open each file in turn to modify content
%-------------------------------------------------------------------------------

\vspace{5em}

Dear Zuckerman Institute team,

I was excited to learn of the current open position for a Senior Staff Associate focusing on development of software for neuroscience applications. I believe my extensive (10+ years) research experience in systems neuroscience (undergraduate, PhD, postdoctoral), as well as my current role as a software engineer developing open source tools (bonsai-rx ecosystem) for neuroscience research and analysis makes me an ideal candidate for the position.

In my current role as a Senior Software Engineer at NeuroGEARS I work as a contributor to the open-source bonsai-rx language and ecosystem. This platform is a popular tool within neuroscience research for building experimental control and data analysis pipelines and contains a suite of packages for applications in computer vision, machine learning, experimental design and many more. Through contributing to this platform I have gained substantial experience in best practices for developing and managing software projects:
\vspace{1em}
\begin{cvitems}
  \item {Git / GitHub version control approaches involving multiple collaborators across different teams}
  \item {Definition of coding standards and application via code review}
  \item {Continuous integration and unit testing of codebases to apply new features and fixes with limited disruption to existing users}
  \item {Generation of auto-documenting code, documentation websites, tutorials and workshops for effective dissemination of new features and libraries to end-users.}
\end{cvitems}
\vspace{1em}

Aside from my development work with NeuroGEARS, I am also heavily involved in workshops and training for users of bonsai-rx and have worked with many labs across the US (including in the Zuckerman Institute) and Europe to deploy bonsai-rx applications for their research requirements. As the user-base of bonsai-rx has grown I have also contributed to the community as a provider of technical support via the discussion forums.

In my previous role as a postdoctoral researcher in the Hires lab at USC I pursued a project using all-optical techniques to study information transfer between cortical layers, using 2-photon imaging and spatial-light modulation for optogenetic targeting. Working with these datasets required advanced data science skills and following effective practices for development and dissemination of data analysis code. During this work I was involved with:
\vspace{1em}
\begin{cvitems}
  \item {Management of git repositories for project data analysis and figure production (Python, MATLAB, jupyter, matplotlib)}
  \item {Utilising dimensionality-reduction techniques to examine evolution of neural population dynamics over task learning}
  \item {Deployment of machine-learning models (tensorflow, Keras) on cloud services (Google cloud) to increase analysis throughput in the lab}
  \item {Development of RNN models to infer functional connectivity in neural networks}
  \item {Organising and teaching workshops on best practices for version control (git) in research projects}
\end{cvitems}
\vspace{1em}

During my PhD research with Andreas Schaefer at the Francis Crick Institute I developed AutonoMouse: a high-throughput platform for mouse behavioural training. This necessitated building new software tools for flexible control of hardware in the system and integration for sensor data. To this end I developed a number of open-source projects including autonomouse-control (Python + PyQt GUI for designing, running and analysing long-term behavioural experiments) and PulseBoy (modular software platform for creating and synchronizing complex digital patterns to drive e.g. solenoid arrays).

Throughout these positions I have gained extensive experience in presentation of research and am highly proficient in visualization and communication of data. These skills have led to several publications in prominent journals (Nature, Neuron, eLife, PLOS ONE, JNeurosci). I am passionate about creating the tools that allow scientists to pursue their research goals effectively, and democratising those tools with dissemination and user-feedback. I believe that this goal, along with my considerable experience in neuroscience research and software engineering, would make me a valuable addition to the Zuckerman Institute team.  

Many thanks for your time and consideration. My resume, along with my personal websites (robodoig.github.io / andrewerskine.uk) include further details on my work, publications and personal projects. I hope to hear from you soon.

Best regards,

Andrew Erskine

%-------------------------------------------------------------------------------
\end{document}
